\chapter{Deployment}\label{ch:deployment}

\section{Data Stores}\label{sec:data-stores}

State in Anubis is somewhat fluid.
Data is parked either in the main Mariadb~\ref{subsec:mairadb} database, or
in the redis cache~\ref{subsec:caching}.

\subsection{Mariadb}\label{subsec:mariadb}

The main data store for Anubis is a \href{https://mariadb.org/}{MariaDB}
\href{https://mariadb.com/kb/en/galera-cluster/}{galera} deployment.
More specifically the
\href{https://github.com/bitnami/charts/tree/master/bitnami/mariadb-galera}{Bitnami mariadb-galera} chart is used.

The advantage of galera is that MariaDB is multi-leader.
Any node of the MariaDB cluster can be read or written to at any time.
In a multi leader database deployment there is a certain tolerance of downed nodes before
service is degraded.
Even if nodes begin to fail, the MariaDB pods that are available can still handle
read and write operations.

All persistent storage in Anubis is parked in MariaDB.
Things like student, course and assignment information are stored here.
Temporary values like autograde results are stored in redis~\ref{subsec:redis}.

\subsection{Redis}\label{subsec:redis}

\href{https://redis.io/}{Redis} in Anubis is where temporary values are stored.
It is assumed that redis is not persistent.
What this means is that the redis deployment should be able to be reset
at any time without critical data loss.
If and when values need to be persisted, MariaDB~\ref{subsec:mariadb} is the better option.

\subsection{Caching}\label{subsec:caching}

Caching in Anubis is handled by the
\href{https://flask-caching.readthedocs.io/en/latest/index.html}{flask-caching} library.
The return values for specific functions can be temporarily stored in Redis for some
period of time.

In the Anubis API there is a great deal of caching.
Almost all view functions that query for database values will cache the results in redis
for some period of time.

Many of the computationally intensive calculation results are cached.
Take the autograde results for example.
To calculate the best submission for a student, all the
calculated submission results for that student and assignment must be pulled
out of the database and examined.
Depending on the student and assignment, this calculation could involve
examining hundreds of submissions searching for the best.
The results for these autograde calculations are then stored in the cache.

For a 3 week window around each assignment deadline the autograde results
for the assignment are periodically calculated.
The purpose of this preprocessing is to ensure that the autograde results
for recent assignments are always available in the cache.
Without this small optimization, the autograde results page in the admin
panel would just about always require 15-30 seconds to load data.

Another example of heavy caching in Anubis would be the public usage visuals.
The visuals themselves are png images that are generated from matplotlib.
To generate the png we must periodically pull out all the submission
and cloud ide session information.
Pulling this data then generating the png can take upwards of 45 seconds to a minute.
45 seconds is obviously an unacceptable amount of time to wait for an image to load
in the browser.
Anubis handles this situation by telling the API to always load the cached png image
from the cache.
The image is then periodically re-generated in the Reaper Cronjob.

Without heavy caching on things like the autograde results and visual generation
the admin panel would be quite slow.

\subsection{RPC Queues}\label{subsec:rpc-queues}

\section{Logging}\label{sec:logging}
\subsection{Filebeat}\label{subsec:filebeat}
\subsection{Elasticsearch}\label{subsec:elasticsearch}

\section{Kubernetes}\label{sec:kubernetes}
\subsection{Helm Chart}\label{subsec:helm-chart}
\subsection{Longhorn}\label{subsec:longhorn}
\subsection{Digital Ocean}\label{subsec:digital-ocean}
\subsubsection{Nodes}\label{subsubsec:digital-ocean-nodes}
\subsubsection{Networking}\label{subsubsec:digital-ocean-networking}

\section{Github}\label{sec:github}
\subsection{Github Classrooms}\label{subsec:github-classrooms}


