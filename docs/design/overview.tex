\chapter{Project Overview}\label{ch:overview}

The Anubis LMS is a tool to give students live feedback from their homework
assignments while they are working on them and before the deadline.
Instead of having students submit a patch file or individual files, each student
will have their own private repo for every assignment.
The way students then submit their work is simply by pushing to their repo before
the deadline.
Students submit as many times as they would like before the deadline.

When a student pushes to their assignment repo, a job is launched in the Anubis cluster.
That job will build their code, and run tests on the results.
Students can then use the live feedback to see which areas they need to improve on
before they get their final grades.

\section{Autograding}\label{sec:autograding}

When a student pushes to their assignment repo, a job is launched in the
Anubis cluster. That job will build their code, and run tests on the results.
Students can then use the live feedback to see which areas they need to improve on
before they get their final grades.


\section{Anubis Cloud IDEs}\label{sec:anubis-cloud-ides}

New in version v2.2.0, there is now the Anubis Cloud IDE. Using some kubernetes magic, we are able to
host \href{https://theia-ide.org/}{theia} servers for individual students. These are essentially VSCode instances
that students can access in the browser. What makes these so powerful is that students can access a terminal
and type commands right into a bash shell which will be run in the remote container. With this setup students
have access to a fully insulated and prebuilt linux environment at a click of a button.


\section{Insights}\label{sec:insights}

Anubis passively captures very interesting usage data from users.
Most users elect to using the Cloud IDEs as they offer an easily accessable environment.
When they do this, the autosave pushes their work to github every 5 minutes, and
submission tests are run on the repo.
With this feedback loop, Anubis captures near minute by minute progress on an assignment
for most all users.
With the usage data generated by Anubis, very interesting questions and answers can be
