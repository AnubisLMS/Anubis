\documentclass[11pt]{article}

\usepackage{times}
\usepackage{alltt}
\usepackage{url}


%% Page layout
\oddsidemargin 0pt
\evensidemargin 0pt
\textheight 600pt
\textwidth 469pt
\setlength{\parindent}{0em}
\setlength{\parskip}{1ex}

\include{config}

\title{15-213 Fall 20xx\\
Lab Assignment L2: Defusing a Binary Bomb\\
Assigned: Sept. 13, Due: Friday Sept. 22\\
}

\author{}
\date{}

\begin{document}

\maketitle

Harry Bovik ({\tt bovik@cs.cmu.edu}) is the lead person for this lab.

\section{Introduction}

The nefarious {\em Dr.~Evil} has planted a slew of ``binary bombs'' on
our class machines.  A binary bomb is a program that consists of a
sequence of phases. Each phase expects you to type a particular string
on \texttt{stdin}.  If you type the correct string, then the phase is
{\em defused} and the bomb proceeds to the next phase. Otherwise, the
bomb {\em explodes} by printing \verb."BOOM!!!". and then
terminating. The bomb is defused when every phase has been defused.

There are too many bombs for us to deal with, so we are giving each
student a bomb to defuse.  Your mission, which you have no choice but to
accept, is to defuse your bomb before the due date.  Good luck, and
welcome to the bomb squad!

\section*{Step 1: Get Your Bomb}

You can obtain your bomb by pointing your Web browser at:

\begin{verbatim}
    http://$Bomblab::SERVER_NAME:$Bomblab::REQUESTD_PORT/
\end{verbatim}

This will display a binary bomb request form for you to fill in.
Enter your user name and email address and hit the Submit button.  The
server will build your bomb and return it to your browser in a
\texttt{tar} file called \texttt{bombk.tar}, where $k$ is the unique
number of your bomb.

Save the \texttt{bombk.tar} file to a (protected) directory
in which you plan to do your work.  Then give the command: 
\verb@tar -xvf bombk.tar@.  This will create a directory called
\texttt{./bombk} with the following files:
\begin{itemize}
\item \texttt{README}: Identifies the bomb and its owners.
\item \texttt{bomb}: The executable binary bomb.
\item \texttt{bomb.c}: Source file with the bomb's main routine and a friendly greeting from Dr. Evil.
\item \texttt{writeup.\{pdf,ps\}}: The lab writeup.
\end{itemize}

If for some reason you request multiple bombs, this is not a
problem. Choose one bomb to work on and delete the rest.

\section*{Step 2: Defuse Your Bomb}
Your job for this lab is to defuse your bomb.

You must do the assignment on one of the class machines.  In fact,
there is a rumor that Dr. Evil really is evil, and the bomb will
always blow up if run elsewhere.  There are several other
tamper-proofing devices built into the bomb as well, or so we hear.

You can use many tools to help you defuse your bomb. Please look at
the {\bf hints} section for some tips and ideas.  The best way is to
use your favorite debugger to step through the disassembled binary.

Each time your bomb explodes it notifies the bomblab server, and you lose
1/2 point (up to a max of 20 points) in the final score for the lab.
So there are consequences to exploding the bomb. You must be careful!

The first four phases are worth 10 points each. Phases 5 and 6 are a
little more difficult, so they are worth 15 points each. So the
maximum score you can get is 70 points.

Although phases get progressively harder to defuse, the expertise you
gain as you move from phase to phase should offset this
difficulty. However, the last phase will challenge even the best
students, so please don't wait until the last minute to start.

The bomb ignores blank input lines. If you run your bomb with a
command line argument, for example, 
\begin{alltt}
    linux> {\em ./bomb psol.txt}
\end{alltt}
then it will read the input lines from {\tt psol.txt} until it reaches
EOF (end of file), and then switch over to {\tt stdin}. In a moment of
weakness, Dr.~Evil added this feature so you don't have to keep
retyping the solutions to phases you have already defused.

To avoid accidentally detonating the bomb, you will need to learn how to
single-step through the assembly code and how to set breakpoints.  You
will also need to learn how to inspect both the registers and the
memory states.  One of the nice side-effects of doing the lab is that
you will get very good at using a debugger.  This is a crucial skill
that will pay big dividends the rest of your career.


\section*{Logistics}

This is an individual project. All handins are electronic.
Clarifications and corrections will be posted on the course message
board.


\section*{Handin}

There is no explicit handin. The bomb will notify your instructor
automatically about your progress as you work on it.  You can keep
track of how you are doing by looking at the class scoreboard at:

\begin{verbatim}
    http://$Bomblab::SERVER_NAME:$Bomblab::REQUESTD_PORT/scoreboard
\end{verbatim}

This web page is updated continuously to show the progress for
each bomb.

\section*{Hints {\em (Please read this!)}}

There are many ways of defusing your bomb.  You can examine it in
great detail without ever running the program, and figure out exactly
what it does.  This is a useful technique, but it not always easy to
do.  You can also run it under a debugger, watch what it does step
by step, and use this information to defuse it.  This is probably the
fastest way of defusing it.

We do make one request, {\em please do not use brute force!}  You could
write a program that will try every possible key to find the right
one.  But this is no good for several reasons:
\begin{itemize}
\item You lose 1/2 point (up to a max of 20 points) 
every time you guess incorrectly and the bomb explodes.  

\item Every time you guess wrong, a message is sent to the bomblab server.  
You could very quickly saturate the network with these messages, and cause
the system administrators to revoke your computer access.

\item We haven't told you how long the strings are, nor have we told you
what characters are in them.  Even if you made the (incorrect) assumptions
that they all are less than 80 characters long and only contain
letters, then you will have $26^{80}$ guesses for each phase.  This will
take a very long time to run, and you will not get the answer before
the assignment is due.
\end{itemize}

There are many tools which are designed to help you figure out both
how programs work, and what is wrong when they don't work.  Here is a
list of some of the tools you may find useful in analyzing your bomb,
and hints on how to use them.

\begin{itemize}

\item {\tt gdb }

The GNU debugger, this is a command line debugger tool available on
virtually every platform.  You can trace through a program line by
line, examine memory and registers, look at both the source code and
assembly code (we are not giving you the source code for most of your
bomb), set breakpoints, set memory watch points, and write scripts.

The CS:APP web site 
\begin{verbatim}
http://csapp.cs.cmu.edu/public/students.html
\end{verbatim}
has a very handy single-page {\tt gdb} summary that you can print out
and use as a reference. Here are some other tips for using {\tt gdb}.
\begin{itemize}

\item To keep the bomb from blowing up every time you type in a wrong
input, you'll want to learn how to set breakpoints.

\item For online documentation, type ``{\tt help}'' at the {\tt gdb} 
command prompt, 
or type ``{\tt man gdb}'', or ``{\tt info gdb}'' at a Unix prompt.  
Some people also
like to run {\tt gdb} under {\tt gdb-mode} in {\tt emacs}.
\end{itemize}

\item \texttt{objdump -t}

This will print out the bomb's symbol table. The symbol table includes the
names of all functions and global variables in the bomb, the names of
all the functions the bomb calls, and their addresses.  You may learn
something by looking at the function names!

\item \texttt{objdump -d}

Use this to disassemble all of the code in the bomb.  You can also
just look at individual functions.  Reading the assembler code can
tell you how the bomb works. 

Although {\tt objdump -d} gives you a lot of information, it doesn't tell
you the whole story. Calls to system-level functions are displayed in
a cryptic form. For example, a call to {\tt sscanf} might appear as:
\begin{verbatim}
8048c36:  e8 99 fc ff ff  call   80488d4 <_init+0x1a0> 
\end{verbatim}
To determine that the call was to {\tt sscanf}, you would need to
disassemble within {\tt gdb}.


\item {\tt strings }

This utility will display the printable strings in your bomb.



\end{itemize}

Looking for a particular tool?  How about documentation?  Don't
forget, the commands {\tt apropos}, {\tt man}, and {\tt info} are your
friends.  In particular, {\tt man ascii} might come in useful. {\tt
info gas} will give you more than you ever wanted to know about the
GNU Assembler. Also, the web may also be a treasure trove of
information.  If you get stumped, feel free to ask your instructor for help.

\end{document}
